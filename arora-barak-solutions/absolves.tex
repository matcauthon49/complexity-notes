% Title:
% 	REPORT
% ----------------------
% Description:
%   A non-academic looking report.
%
% Creator: Tommy O.

% -------------------------------------------------------------------------
% Package imports
% -------------------------------------------------------------------------
\documentclass[12pt, a4paper, twoside]{article}% 'twocolumn' 
\usepackage[utf8]{inputenc}% Allow input to be UTF-8
\usepackage{babel}% Alternative: 'norsk'
\usepackage{graphicx}% For importing graphics
\usepackage{amsthm, amsfonts, amssymb}% All the AMS packages
\usepackage{mathtools}% Fixes a few AMS bugs
\usepackage[expansion=false]{microtype}% Fixes to make typography better
\usepackage{hyperref}% For \href{URL}{text}
\usepackage{fancyhdr}% For fancy headers
\usepackage[sharp]{easylist}% Easy nested lists
\usepackage{parskip}% Web-like paragraphs
\usepackage{multicol}% For multiple columns
\usepackage[linesnumbered,ruled]{algorithm2e}% For algorithms
\usepackage{tikz-cd}% For diagrams
\usepackage{listings}% To include source-code
\usepackage{booktabs}% Professional tables
%\usepackage{txfonts}% Font which looks less 'academic'
%\usepackage[sc]{mathpazo}% A nice font, alternative to CM
%\usepackage[headings]{fullpage}% Make margins smaller
\usepackage[a4paper, margin = 3.175cm, includehead, includefoot]{geometry}% May be used to set margins

\usepackage{blindtext}% For dummy text
% -------------------------------------------------------------------------
% Package setup
% -------------------------------------------------------------------------

% Setup for the fancyhdr package
\renewcommand{\headrulewidth}{0pt}
\renewcommand{\footrulewidth}{0.4pt}
\rhead[]{\nouppercase{\leftmark}}
\lhead[\nouppercase{\leftmark}]{}
\chead[]{}
\rfoot[]{\mytitle \hspace{1em} \thepage}
\lfoot[\thepage \hspace{1em} \mytitle]{}
\cfoot[]{}

% Section numbers in equations
\numberwithin{equation}{section} 

% -------------------------------------------------------------------------
% Misc settings
% -------------------------------------------------------------------------

% Spacing between easylist items
\newcommand{\listSpace}{-0.25em}

% If 'twocolumn', set the spacing as such
\setlength{\columnsep}{1cm}

\newcommand{\np}{\mathbf{NP}}
\newcommand{\p}{\mathbf{P}}
\newcommand{\ip}{\mathbf{IP}}
\newcommand{\pspace}{\mathbf{PSPACE}}
\newcommand{\am}{\mathbf{AM}}
\newcommand{\ma}{\mathbf{MA}}
\newcommand{\F}{\mathbb{F}}
\newcommand{\E}{\mathbb{E}}
\newcommand{\rel}{\mathcal{R}}
\newcommand{\lang}{\mathcal{L}}
\newcommand{\query}{\mathcal{Q}}
\newcommand{\prover}{\mathcal{P}}
\newcommand{\verifier}{\mathcal{V}}
\newcommand{\decision}{\mathcal{D}}
\newcommand{\simulator}{\mathcal{S}}
\newcommand{\sat}{\mathsf{SAT}}
\newcommand{\nc}{\mathbf{NC}}
\newcommand{\hvzk}{\mathbf{HVZK}}
\newcommand{\zk}{\mathbf{ZK}}
\newcommand{\pzk}{\mathbf{PZK}}
\newcommand{\czk}{\mathbf{CZK}}
\newcommand{\szk}{\mathbf{SZK}}
\newcommand{\pc}{\mathsf{pc}}
\newcommand{\vc}{\mathsf{vc}}
\newcommand{\vr}{\mathsf{vr}}
\newcommand{\poly}{\mathsf{poly}}
\newcommand{\add}{\mathsf{add}}
\newcommand{\mul}{\mathsf{mul}}
\newcommand{\view}{\mathsf{View}}
\newcommand{\trace}{\textsc{Trace}}
\newcommand{\bpp}{\mathbf{BPP}}
\newcommand{\hilbert}{\mathcal{H}}
\newcommand{\ket}[1]{\left|#1\right\rangle}
\newcommand{\bra}[1]{\left\langle#1\right|}
\newcommand{\braket}[2]{\langle#1|#2\rangle}
\newcommand{\puncture}{\mathsf{Puncture}}

% -------------------------------------------------------------------------
% Document variables
% -------------------------------------------------------------------------

\def\mytitle{\textbf{Solutions to Computational Complexity}}
\title{\mytitle}
\author{Naman Kumar}

% -------------------------------------------------------------------------
% Document start
% -------------------------------------------------------------------------
\begin{document}
	\maketitle
	\pagestyle{fancy}
	\begin{abstract}
		 The following document contains solutions to the book `Computational Complexity' by Sanjeev Arora and Boaz Barak. Some trivial problems have been omitted.
	\end{abstract}

	\newpage
	\tableofcontents
	\newpage
	
	% -------------------------------------------------------------------------
	% Document content start
	% -------------------------------------------------------------------------
	
	\section{$\np$ and $\np$-Completeness}
	\label{sec:introduction}
	\subsection{Review}
\begin{itemize}
	\item The binomial theorem and binomial coefficients provide a way to count objects.
	\item Exact values of binomial coefficients are hard to compute, but they can be roughly estimated. Consider the following inequalities:
	\begin{enumerate}
		\item $1+t<e^t$ for $t\neq 0$
		\item $1-t>e^{-t-t^2/2}$ for $0<t<1$
	\end{enumerate}
	they can be used to prove 
	\begin{align*}
		\left(\frac{n}{k}\right)^k\leq\binom{n}{k}&\hspace{10mm} \sum_{i=1}^k\binom{n}{i}\leq \left(\frac{en}{k}\right)^k.
	\end{align*}
	\item The Stirling approximation is another approximation which shows
	$$n!\approx\left(\frac{n}{e}\right)^n\sqrt{2\pi n}e^{\alpha_n}$$
	with $1/(12n+1)<\alpha_n<1/12n$. Similarly, an asymptotic approximation for the $k$-factorial is
	$$(n)_k=n^ke^{-k^2/2n-k^3/6n^2+o(1)}$$ which is valid for $k=o(n^{3/4}).$
	\item The stars and bars technique allows you to partition a set into ordered partitions.
	\item Double counting can be done using incidence matrices: in this technique you count the rows and the columns of a zero-one matrix, which should yield the same sums.
	\item The number of vertices in a graph with odd degree is even.
	\item Jensen's Inequality: a function is convex if 
	$$f(\lambda a+(1-\lambda)b)\leq \lambda f(a)+(1-\lambda)f(b)$$
	for $0\leq\lambda\leq 1$. Suppose that each $\lambda_i$ is in $[0,1]$, and $\sum\lambda_i=1$. If $f$ is convex then
	$$f\left(\sum_{i=1}^n\lambda_ix_i\right)\leq\sum_{i=1}^n\lambda_if(x_i).$$
	\item Inclusion-exclusion provides a way to sum up all the elements in some sets that intersect.
	\item Derangements are permutations which don't fix any points.
\end{itemize}

\subsection{Problems}

\paragraph{In-text.} Prove:
\begin{align}
	&\sum_{x\in Y} d(x)=\sum_{A\in\mathcal{F}}|Y\cap A| \text{ for any } Y\subseteq X.\\
	&\sum_{x\in X}d(x)^2=\sum_{x\in\mathcal{F}}\sum_{x\in A}d(x)=\sum_{A\in\mathcal{F}}\sum_{B\in\mathcal{F}}|A\cap B|.
\end{align}

\paragraph{Answer.} We can proceed by a counting argument. For the first part, consider the incidence matrix $M=(m_{x,a})$ of $\mathcal{F}$. Adding the rows belonging to the elements of $x\in Y$ gives us the left hand side. If we `slice' the matrix in this way, however (removing all the rows belonging to elements $x\notin Y$) then summing via the columns we get the sum of only those elements which are both in $A$ and in $Y$, i.e. $|A\cap Y|$.

For the second part, it is enough to notice that if we replace each entry $1$ in the incidence matrix with $d(x)$, then adding along the rows gives $\sum_{x\in X}d(x)^2$ while adding along the columns gives the term in the central equality (we add $d(x)$ for every $x\in A$, and then we sum over each of the $A$s). The final equality follows from noticing that
\begin{align*}
	\sum_{x\in\mathcal{F}}\sum_{x\in A}d(x)&=\sum_{A'\in\mathcal{F}}\left(\sum_{x\in A}|A\cap A'|\right)\\
	&=\sum_{A\in\mathcal{F}}\sum_{B\in\mathcal{F}}|A\cap B|.
\end{align*}
where the second equality follows from substituting the first part.


\begin{center}
	\line(1,0){70}
\end{center}

\paragraph{Question 1.1} In how many ways can we distribute $k$ balls to $n$ boxes so that each box has at most one ball?

\paragraph{Answer.} Depends on whether $k\leq n$ or $k>n$. If $k\leq n$, we can choose $\binom{n}{k}$ boxes, and the balls can be ordered in $k!$ ways. it follows that the number of ways is $\binom{n}{k}k!=(n)_k$.

If $k>n$, then there are $\binom{k}{n}$ ways we can choose the balls, and we can order them in $n!$ ways. This gives us the answer of $(k)_n$.

\begin{center}
	\line(1,0){70}
\end{center}

\paragraph{Question 1.2} Show that for every $k$ the product of any $k$ consecutive natural numbers is divisible by $k!$.

\paragraph{Answer.} Let the $k$ numbers be $n+1,\dots,n+k$. Consider the number $\binom{n+k}{k}$. This is the number of ways you can choose $k$ numbers from $n+k$ numbers, and is an integer. Expanding, we get
$$\binom{n+k}{k}=\frac{n!}{k!(n-k)!}$$
which clearly shows that $k!$ divides $(n+1)(n+2)\dots(n+k)$.

\begin{center}
	\line(1,0){70}
\end{center}

\paragraph{Question 1.3} Show that the number of pairs $(A,B)$ of distinct subsets of $\{1,\dots,n\}$ with $A\subset B$ is $3^n-2^n$.

\paragraph{Answer.} We can proceed as follows. Select a subset $S$ of size $k$, and then select a subset of $S$. The former can be done in $\binom{n}{k}$ ways, while the latter can be done in $2^k-1$ ways. This gives us the sum
$$\sum_{k=0}^n\binom{n}{k}(2^k-1)$$
The binomial theorem tells us that this is equal to $(2+1)^n-(1+1)^n=3^n-2^n$.

\begin{center}
	\line(1,0){70}
\end{center}

\paragraph{Question 1.4} Show that
$$\binom{n}{k}=\frac{n}{k}\binom{n-1}{k-1}.$$

\paragraph{Answer.} We will count the number of ways to choose $k$ balls from $n$ balls. Just choosing the $k$ balls is the left hand side. Alternately, we can choose one ball first, and then choose $k-1$ balls from the remaining $n-1$ balls. However, this leads to a $k$-recounting, since each $k$ size subset is selected $k$ times (once when each element is the `fixed' element). We are done.

\begin{center}
	\line(1,0){70}
\end{center}

\paragraph{Question 1.6} There is a set of $2n$ people: $n$ male and $n$ female. A good party is a set with the same number of male and female. How many possibilities are there to build such a good party?

\paragraph{Answer.} For each $k$ there are $k$ ways to choose men and $k$ ways to choose women, so the total number of parties of size $k$ are $\binom{n}{k}^2$. Adding over all the $k$'s we get
$$\sum_{i=0}^{n}\binom{n}{i}^2.$$

\begin{center}
	\line(1,0){70}
\end{center}

\paragraph{Question 1.7}
Use Proposition 1.3 to show that
$$\sum_{i=0}^r\binom{n+i-1}{i}=\binom{n+r}{r}$$

\paragraph{Answer.} We expand the RHS using Pascal's identity, then recursively expand one of the terms.

\begin{center}
	\line(1,0){70}
\end{center}

\paragraph{Question 1.8} Let $0\leq a\leq m\leq n$ be integers. Show that
$$\sum_{i=m}^n\binom{i}{a}=\binom{n+1}{a+1}-\binom{m}{a+1}.$$

\paragraph{Answer.} Same trick as the previous question, expand $\binom{n+1}{a+1}$.

 \begin{center}
 	\line(1,0){70}
 \end{center}

\paragraph{Question 1.9} Prove the Cauchy-Vandermonde identity:
$$\binom{p+q}{k}=\sum_{i=0}^k\binom{p}{i}\binom{q}{k-i}.$$

\paragraph{Answer.} We count twice. The left is selecting $k$ items from $p+q$ items. Another way we can count this is to select $i$ items from $p$ items and $k-i$ items from $q$ items, then add over every $i$.

\begin{center}
	\line(1,0){70}
\end{center}

\paragraph{Question 1.10} Show that
$$\sum_{k=0}^n\binom{n}{k}^2=\binom{2n}{n}.$$

\paragraph{Answer.} Start with the previous problem, then set $p=q=k=n$. It follows that
$$\binom{2n}{n}=\sum_{k=0}^n\binom{n}{k}\binom{n}{n-k}.$$ However, we know that to choose $k$ things, we can choose $k$ or choose $n-k$, so the latter terms are equal.

\begin{center}
	\line(1,0){70}
\end{center}

\paragraph{Question 1.11} Prove the following analogy of the binomial theorem for factorials:
$$(x+y)_n = \sum_{k=0}^n\binom{n}{k}(x)_k(y)_{n-k}.$$

\paragraph{Answer.} Immediately follows from the same consideration as the typical proof of the binomial theorem.

\begin{center}
	\line(1,0){70}
\end{center}

\paragraph{Question 1.28} This took me way too fucking long. Here's how you do it:

First, let $|E|$ be the number of edges in the graph. Note that $|B|D\geq|E|\geq|A|d$ by definition, since the \textit{minimum} number of edges leaving $A$ is $|A|d$, and the \textit{maximum} number of edges leaving $B$ is $|B|D$. Together with the fact given in the question this gives $|A|d=|B|D$. Furthermore, because the number of edges is sandwiched between these two, quantities, we know that $E=|A|d$. This tells us that every vertex in $A$ \textit{must} have degree $d$. Now take $B_0$ to be the set such that every neighbor has $\alpha D/2$ vertices to $A_0$.

It follows that the number of edges leaving $A_0$ is $|A_0|d = \alpha|A|d = \alpha|B|D$. Of these, some go to $B_0$, while others go outside of $B_0$. It follows that $|E_{A_0}|=|E_{A_0\rightarrow B_0}|+|E_{A_0\rightarrow B\setminus B_0}|$. We know that $|E_{A_0\rightarrow B_0}|\leq \alpha|B_0|D/2$. Suppose that $|E_{A_0\rightarrow B\setminus B_0}|\geq|E_{A_0\rightarrow B_0}|$, which implies $|E_{A_0}|=\alpha|B|D\leq \alpha|B_0|D/2+\alpha|B_0|D/2=\alpha|B_0|D$, which is preposterous since $|B_0|\leq |B|$. This proves (iii), that $|E_{A_0\rightarrow B\setminus B_0}|\leq |E_{A_0\rightarrow B\setminus B_0}|$, which implies that more than half of the edges leaving $A_0$ go to $B_0$.

The first part now easily follows. The number of edges going from $A_0$ to $B_0$ is exactly the number of edges going from $B_0$ to $A_0$, which we know is greater than $\alpha|B|D/2$ (which is half the number of edges leaving $A_0$). The edges leaving $B_0$ include this number \textit{and more}. This means that $|B_0|D\geq \alpha|B|D/2$, which gives us (i), ie. $|B_0|\geq\alpha|B|/2$.

Very annoying problem. It's not conceptually difficult but there's a number of very shitty moving parts. Happy I solved it, finally.

\begin{center}
	\line(1,0){70}
\end{center}

\paragraph{Question 1.29} This was also mildly annoying, but I got it after a few minutes of serious thought. You apply Jensen's inequality to the parameters $f(x)=x^{t/s}$, $\lambda_i = 1/n$, and $x_i = a_i^{s}$. The inequality follows from the fact that $f$ is convex iff $t\geq s$.

\begin{center}
	\line(1,0){70}
\end{center}

\paragraph{Question 1.37} This is not interesting, but is used later in the book. Really the only thing here is that this is in some sense `partial' Inclusion-Exclusion: it's written in a misleading way, but it's actually just inclusion-exclusion with some of the terms chopped off.

In the odd cases, the opposite will happen: it'll be greater for odd $k$ and lesser for even $k$.


	
	\newpage
	\section{Diagonalization}
	\label{sec:diagonalization}
	\paragraph{Question 3.1} Show that the following language is undecidable:
\begin{equation*}
	\left\{\lfloor M\rfloor: M \text{ is a machine that runs in } 100n^2+200 \text{ time}\right\}
\end{equation*}
\paragraph{Answer.} We use simple diagonalization. Suppose to the contrary that this language is decidable; define $M$ to be the TM which outputs $1^{100|x|^2+201}$ if $M_x$ runs in time $100|x|^2+200$ on input $x$ and $0$ otherwise. Now let $M=M_i$ for some $i$ in a universal encoding of Turing Machines, and we run $M$ on $i$. Clearly this leads to a contradiction because $M_i$ must print $0$ if $M_i$ runs for time $100i^2+201$ and vice-versa. So this other language is decidable. But any machine which decides $L$ can decide the other language, which proves that $L$ must be undecidable as well.

\begin{center}
	\line(1,0){70}
\end{center}

\paragraph{Question 3.2} Show that $\mathbf{SPACE}(n)\neq\np$. (Note that we do not know if either class is contained in the other.)

\paragraph{Answer.} We will use here \textit{space hierarchy} theorem, which states that for all time-constructible $f(x)$ and $g(x)$ such that $f(x)=o(g(x))$, 
\begin{equation*}
	\mathbf{SPACE}(f(x))\subsetneq\mathbf{SPACE}(g(x)).
\end{equation*}

Our proof strategy will be as follows: we will show that the class $\np$ is closed under polynomial-time reductions, and that $\mathbf{SPACE}(n)$ is not. Since these two classes have different closure properties, we can conclude they are not the same class.

To show the former, assume that $L_1\in\np$ and that $L_2$ is polynomial-time reducible to $L_1$. Then an $\np$-algorithm for $L_2$ is to first deterministically reduce it to $L_1$ (which can be done in polynomial time), and then run the appropriate $L_1$-sequence. This proves that $L_2\in\np$ and that $\np$ is closed under polynomial-time reductions.

To show that $\mathbf{SPACE}(n)$ is not closed under polynomial-time reductions, just notice that any language $\mathbf{SPACE}(n^2)$ can be reduced to $\mathbf{SPACE}(n)$ by padding an instance $|x|$ with $1^{|x|^2}$, which can be done in polynomial time. However, we know by the space hierarchy theorem that $\mathbf{SPACE}(n)\subsetneq\mathbf{SPACE}(n^2)$, and so we can conclude that that $\mathbf{SPACE}(n)$ is not closed under polytime reductions, so we are done.

\begin{center}
	\line(1,0){70}
\end{center}

\paragraph{Question 3.3} Show that there is a language $B\in\mathbf{EXP}$ such that $\np^B\neq\p^B$.

\paragraph{Answer.} The language $B$ in the Baker-Gill-Solovay theorem is in $\mathbf{EXP}$, since it can be decided by a machine that runs $M_i$ on the input for $2^n/10$ steps, which is allowed by $\mathbf{EXP}$.

\begin{center}
	\line(1,0){70}
\end{center}

\paragraph{Question 3.4} Say that a class $C_1$ is \textit{superior} to a class $C_2$ if there is a machine $M_1$ in class $C_1$ such that for every machine $M_2$ in class $C_2$ and every large enough $n$ there is an input of size between $n$ and $n^2$ on which $M_1$ and $M_2$ answer differently.
\begin{enumerate}
	\item[(a)] Is $\mathbf{DTIME}(n^{1.1})$ superior to $\mathbf{DTIME}(n)$?
	\item[(b)] Why does our proof of the Nondeterministic Hierarchy Theorem not prove $\mathbf{NTIME}(n^{1.1})$ superior to $\mathbf{NTIME}(n)$?
\end{enumerate}

\paragraph{Answer.}
\begin{enumerate}
	\item[(a)] Yes. Consider the function $f(x)$ which takes $1\mapsto 1$, $2\mapsto 1$, $3\mapsto 2$, $4\mapsto 1$, $5\mapsto 2$, $6\mapsto 3$, $7\mapsto 1$ and so on. This function is computable in polynomial time and forms a counting of TMs such that for TM there is a large enough $n$ such that the description of the TM appears between $n$ and $n^2$ for all TMs. Then run $D$ given in the proof of the deterministic time-hierarchy theorem on this TM labling instead. It follows that $\mathbf{DTIME}(n^{1.1})$ is superior to $\mathbf{DTIME}(n)$.
	\item[(b)] The proof does not immediately work because $f(i+1)>f(i)^2$ for large enough $i$.
\end{enumerate}

\begin{center}
	\line(1,0){70}
\end{center}

\paragraph{Question 3.5} Show that there exists a function that is not time-constructible.

\paragraph{Answer.} Consider a numbering of Turing Machines and define the function $f$ such that
$$f(1^x)=\begin{cases}
	1\text{ if $M_x(x)$ works for $0$ steps or longer than $1$ step}\\
	0\text{ otherwise.}
\end{cases}$$

Clearly there exists no Turing machine which computes this function, since it disagrees with every Turing Machine $M_x$ on the input $1^x$. Hence the function is not time-constructible.

\begin{center}
	\line(1,0){70}
\end{center}

\paragraph{Question 3.6}
\begin{itemize}
	\item[(a)] Prove that the function $H$ defined in the proof of theorem $3.3$ is computable in polynomial time.
	\item[(b)] Let $\mathsf{SAT}_H$ be defined as in the proof of Theorem $3.3$ for a polynomial-time computable function $H:\mathbb{N}\rightarrow\mathbb{N}$ such that $\lim_{n\rightarrow\infty}H(n)=\infty.$ Prove that if $\sat_H$ is $\np$-complete, then $\sat$ is in $\p$.
\end{itemize} 

\paragraph{Answer.} \begin{itemize}
	\item[(a)] We can incrementally compute $H(n)$ by starting from $i=1$ and proceeding to $i\leq\log n$. To do this we need to simulate $\log\log n$ machines on inputs of length $\log n$, where the number of steps is at most $o(n)$. Furthermore, we need to compute $\sat$ on these imputs, but this can be done simply by brute force since the size of the $\sat$ instance is at most $\log n$. The entire procedure is polynomial time.
	
	\item[(b)] Suppose that $\sat_H$ is $\np$-complete. Then there is a polytime reduction from $\sat$ to $\sat_H$. Let the time complexity of this reduction be $p(|x|)$ for some polynomial $p$, and let the monomial corresponding to the leading coefficient of $p$ be $n^c$ for some constant $c$. Since $H(n)$ grows faster than any constant, we take instances $n$ of size such that $H(n)>2^c$. Now consider a $\sat$ instance of the size $n$. By the time complexity of the reduction, we can reduce it to a $\sat_H$ instance of length at most $n^c$.
	
	The size of a $\sat_H$ instance is $|x|+|x|^{H(|x|)}$, where $|x|$ is the size of the $\sat$ instance encoded in the $\sat_H$ instance. Suppose that the size of the $\sat$ instance is greater than  $n^{c/2^c}$. Then the total size of the $\sat_H$ instance will be $>{n}^{c}$. Thus, the size of the $\sat$ instance must decrease with each iteration of reduction. It follows that if we `hard-code' the solution to all $\sat$ instances in a TM up till the first $n$ such that $H(n)>2^c$, we can solve all the others in polynomial time using downward self-reducibility. We are done. 
\end{itemize}
	
	



	\bibliographystyle{apalike} % 'alpha' is also good
	%\bibliography{bibliography} % Reference to 'bibliography.bib'
\end{document}