\section{Oblivious Transfer}

Oblivious Transfer is a simple MPC functionality parametrized by a selection of sender inputs and a receiver index. 

\begin{figure}[h]
	\begin{mdframed}[
		linecolor=black,
		linewidth=1pt,
		roundcorner=5pt,
		backgroundcolor=white,
		userdefinedwidth=\textwidth,
		]
		\vspace{2mm}
		\textbf{Parameters:} The sender $S$ has a selection of $N$ input strings $(m_0,\dots,m_{N-1})$, while the receiver $R$ has an index $i\in[N]$.\\
		\textbf{Outputs:} $S$ receives nothing while $R$ receives $m_i$.
		\vspace{2mm}
	\end{mdframed}
	\caption{$1$-out-of-$N$ Oblivious Transfer.}
	\label{fig:OT}
\end{figure}

\subsection{Protocol for $1$-out-of-$2$ OT}

We now demonstrate a simple protocol for $1$-out-of-$2$ OT \cite{esgot}. We begin with a CPA-secure encryption scheme $(\mathsf{Gen},\mathsf{Enc},\mathsf{Dec})$, and define the notion of oblivious sampling.

\begin{definition}[PKE with Obliviously Sampleable Encryption Key]
	An \rarrow encryption scheme $(\mathsf{Gen},\mathsf{Enc},\mathsf{Dec})$ has obliviously sampleable public keys if there exist algorithms $\mathsf{Samp}$ and $\mathsf{pkSamp}$ such that 
	\begin{itemize}
		\item $\{\mathsf{Samp}(1^\lambda)\}$ is computationally indistinguishable from $\{\pk:(\pk,\sk)\rgets\mathsf{Gen}(1^\lambda)\}$.
		\item $\{(\pk,r):r\rgets\{0,1\}^\secpar, \pk\leftarrow\mathsf{Samp}(1^\secpar;r)\}$ is computationally indistinguishable from $\{(\pk,r):(\pk,\sk)\rgets\gen, r\gets\mathsf{pkSim}(\pk)\}$.
	\end{itemize}
\end{definition}

The protocol proceeds as follows.

\begin{figure}[h]
	\begin{mdframed}[
		linecolor=black,
		linewidth=1pt,
		roundcorner=5pt,
		backgroundcolor=white,
		userdefinedwidth=\textwidth,
		]
		\vspace{2mm}
		\begin{itemize}
			\item Receiver runs $(\pk,\sk)\gets\gen(1^\secpar)$ and $\pk'\gets\mathsf{Samp}(1^\secpar)$ and sets $\pk_b=\pk$, and $\pk_{1-b}=\pk'$.
			\item Receiver sends $\pk_0$ and $\pk_1$.
			\item Sender encrypts $c_i=\enc_{\pk_i}(m_i)$.
			\item Sender sends $c_0, c_1$.
			\item Receiver decrypts $m_b = \dec_{\sk_b}(c_b)$.
		\end{itemize}
		\vspace{2mm}
	\end{mdframed}
	\caption{$1$-out-of-$2$ Oblivious Transfer from obliviously sampleable PKE.}
	\label{fig:piot}
\end{figure}

\subsection{Assumptions Underlying OT}

Unfortunately, it is not known how oblivious transfer can be constructed from symmetric-key primitives. In particular, we can show that black-box constructions of oblivious transfer from OWFs is impossible.

\begin{theorem}[Black-Box Separation of OT and OWFs]
	There \rarrow is no construction of Oblivious Transfer that makes black-box use of One-Way Functions.
\end{theorem}

\begin{proof}
	We will show that Any Oblivious Transfer protocol implies a Key Exchange Protocol.  The result then follows from \cite{impagliazzo1989limits}, which separates key exchange from all one-way functions.
	
	The KE protocol is very simple:
	\begin{enumerate}
		\item $P_0$ samples $r\rgets\{0,1\}^\secpar$ and sends $(r, 0^\secpar)$ to $\func_\mathsf{OT}$.
		\item $P_1$ sends $0$ to $\func_\mathsf{OT}$.
		\item $P_1$ receives $r$.
	\end{enumerate}

	We claim that this is a secure key exchange. The proof proceeds by hybrid argument: first, the above protocol is indistinguishable to an outside observer from a hybrid in which $P_1$ inputs $1$ instead of $0$ by receiver security, and secondly it is indistinguishable from a hybrid in which $P_0$ inputs $(0^\lambda,0^\lambda)$ instead by sender security. However, this second hybrid is (information-theoretically) independent of $r$, and it follows that the eavesdropper cannot gain any information about $r$, and we are done.
\end{proof}

\subsection{OT in the Correlated Randomness Model}

We now consider the correlated randomness model, where at the beginning of the protocol both $P_0$ and $P_1$ are given random strings $r_0$ and $r_1$ by a trusted party such that they satisfy some correlation. In particular, we will show how to obtain OT from a set of \textit{random OTs.}

\begin{definition}[Random OTs]
	We \rarrow say that two parties $P_0$ and $P_1$ possess random OT correlations if party $P_0$ (the sender) possesses $(\tilde{m}_0, \tilde{m}_1)\rgets\{0,1\}^{2\secpar}$ and party $P_1$ (the receiver) possesses $(\vec{b},(\tilde{m}_{b_i})_{i\in\secpar})\in\{0,1\}^{2\secpar}$ where $\vec{b}\rgets\{0,1\}^\secpar$.
\end{definition}

The above definition formalizes this idea of $P_0$ and $P_1$ having `random OT correlations,' i.e. $P_0$ has random sender inputs and $P_1$ has random receiver inputs and outputs. We now see how this can be utilized to get OT.

\begin{figure}[h]
	\begin{mdframed}[
		linecolor=black,
		linewidth=1pt,
		roundcorner=5pt,
		backgroundcolor=white,
		userdefinedwidth=\textwidth,
		]
		\vspace{2mm}
		\textbf{Parameters:} $P_0$ has input bits $(m_0, m_1)$, while $P_1$ has input bit $s$.\\
		\textbf{Correlations:} $P_0$ has random bits $(\tilde{m}_0,\tilde{m}_1)$, while $P_1$ has random bits $(b, \tilde{m}_b)$.\\
		
		\textbf{Protocol:}
		\begin{enumerate}
			\item $P_1$ sends $b\oplus s$ to $P_0$.
			\item $P_0$ replies with $(r_0, r_1):=(m_{0}\oplus\tilde{m}_{b\oplus s}, m_{1}\oplus\tilde{m}_{1\oplus b\oplus s}).$
		\end{enumerate}
		\vspace{2mm}
	\end{mdframed}
	\caption{OT from Random OT.}
	\label{fig:rot}
\end{figure}

Correctness of the above protocol follows from simple considerations: if $s=0$, then $b\oplus s = b$, and in this case $P_1$ can decrypt the string $r_0$ since it knows $\tilde{m}_b$. If $s=1$, then $b\oplus s = b\oplus 1$, and in this case $P_1$ can decrypt $r_1$ since it knows $\tilde{m}_{b\oplus s\oplus 1}=\tilde{m}_b$.