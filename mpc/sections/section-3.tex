\section{Modifications to OT}

We first construct two variants of OT that provide the additional property of \textit{information-theoretic} security against the receiver.

\subsection{Information-Theoretically Secure OT}

\begin{figure}[h]
	\begin{mdframed}[
		linecolor=black,
		linewidth=1pt,
		roundcorner=5pt,
		backgroundcolor=white,
		userdefinedwidth=\textwidth,
		]
		\vspace{2mm}
		\begin{itemize}
			\item Sender samples $(\pk,\sk)\gets\gen(1^n)$ and sends $\pk$ to receiver.
			\item Receiver samples $s_0, s_1\gets\{0,1\}$ and sets $c_b = \enc_\pk(s_b)$ and $c_{1-b}\gets\mathsf{Samp}$, and sends $c_0$, $c_1$ to sender.
			\item Sender sets $s_i=\dec_\sk(c_i)$ and sends $x_i\oplus s_i$.
			\item Receiver computes
			$$x_b=s_b\oplus (s_b\oplus x_b).$$
		\end{itemize}
		\vspace{2mm}
	\end{mdframed}
	\caption{Variant of OT.}
	\label{fig:VOTPLUS}
\end{figure}

Informally, the security against sender is dependent on the (computational) indistinguishability of determining $c_{1-b}$ from an honestly sampled encryption and on the security of the encryption scheme, while the security against the receiver is information-theoretic since it cannot determine $s_1$. 

We see another variant of OT, secure under the DDH assumption.

\begin{figure}[h]
	\begin{mdframed}[
		linecolor=black,
		linewidth=1pt,
		roundcorner=5pt,
		backgroundcolor=white,
		userdefinedwidth=\textwidth,
		]
		\vspace{2mm}
		\begin{itemize}
			\item Receiver samples $r\gets\mathbb{Z}_q$ and sets $(h_0, h_1)=(g_0^r, g_1^{r+b})$. It sends $(h_0, h_1)$.
			\item Sender samples $a_0, b_0, a_1, b_1\gets\mathbb{Z}_q$ and sets $$c_i = (g_0^{a_i}g_1^{b_i}, h_0^{a_i}h_1^{b_1}/(i\cdot g_1^{b_1})x_i)$$ and sends $(c_0, c_1)$.
			\item Receiver parses $c_b$ as $(c^1, c^2)$ and computes $x_b = c^2/(c^1)^r$.
		\end{itemize}
		\vspace{2mm}
	\end{mdframed}
	\caption{Variant of OT secure assuming DDH.}
	\label{fig:DDHOT}
\end{figure}

\subsection{Dual-Mode Cryptosystem} A dual-mode cryptosystem serves as a generic `toggling' mechanism to achieve information-theoretic OT against the sender or the receiver. A strict definition of dual-mode cryptosystem is given in \cite{dualmode}, along with a generic technique that realizes the OT functionality.

Intuitively, a dual-mode cryptosystem is a hybrid cryptosystem that has two modes: a messy mode and a decryption mode. It has several features:

\begin{enumerate}
	\item The cryptosystem is initialized with a trusted setup that produces a pair $(\crs, t)$ where $\crs$ is the common random string and $t$ is the trapdoor.
	\item Whether the system is in messy or decryption mode is decided during setup. Given a $\crs$, the adversary cannot tell which mode the system is operating in.
	\item There are two encryption branches. $\KeyGen$ takes the branch $\sigma\in\{0,1\}$ as input, and produces a general $\pk$ and an $\sk$ that corresponds to the particular branch.
	\item During encryption, a branch $b$ is specified as input. $b$ could be or could not be $\sigma$.
	\item\textbf{Messy Mode:} In this mode if branch $b\neq\sigma$, an encryption is lossy; all information about the plaintext is lost (if encrypted on $b=\sigma$, then it is a normal encryption). Given $t$, it is possible to obtain the $\sigma$ that corresponds to $\pk$. 
	\item\textbf{Decryption Mode:} In this mode, both branches are decryptable; however, you need the trapdoor to find a key tuple $(\pk,\sk_0,\sk_1)$ which allows you to decrypt. Furthermore, $(\pk,\sk_\sigma)$ are statistically indistinguishable from $\KeyGen(\sigma)$.
\end{enumerate}

We can now proceed with a formalism.

\begin{definition}[Dual Mode Cryptosystem]
	A \rarrow dual-mode cryptosystem with message space $\{0,1\}^\ell$ consists of a tuple of probabilistic algorithms $(\mathsf{Setup},\KeyGen, \enc, \dec,\\ \mathsf{FindMessy},\mathsf{TrapKeyGen})$ that have the following properties.
	\begin{itemize}
		\item $\mathsf{Setup}(1^\secpar, \mu)\rightarrow (\crs, t)$. $\mu$ refers to the mode.
		\item $\KeyGen(\sigma)\rightarrow(\pk,\sk)$. If in messy mode, $\sk$ corresponds to the branch $\sigma$.
		\item $\enc(\pk,b,m)\rightarrow c$. $b$ is the branch.
		\item $\dec(\sk, c)\rightarrow m$.
		\item $\mathsf{FindMessy}(t,\pk)\rightarrow b$. Given the trapdoor (and a possibly malformed public key), the output is the messy branch of $\pk$.
		\item $\mathsf{TrapKeyGen}(t)\rightarrow(\pk,\sk_0,\sk_1)$. Outputs a key tuple which allows you to encrypt/decrypt on both branches.
	\end{itemize}
\end{definition}

\subsubsection{OT From Dual-Mode Cryptosystem}

We now consider the OT of \cite{dualmode}, which almost trivially follows from the definition of the DMC. The protocol can be set up in either mode. In messy mode, the receiver security is computational and the sender is statistical; in decryption mode, the security properties are reversed.

\begin{figure}[h]
	\begin{mdframed}[
		linecolor=black,
		linewidth=1pt,
		roundcorner=5pt,
		backgroundcolor=white,
		userdefinedwidth=\textwidth,
		]
		\vspace{2mm}
		\textbf{Parameters:} $S$ has input $(m_0, m_1)\in\{0,1\}^\ell$, while receiver $R$ has input $\sigma\in\{0,1\}$. Suppose that the system is setup in mode $\mu$.\\
		
		\textbf{Protocol:}
		\begin{enumerate}
			\item Both $S$ and $R$ query $\func^{\mu}_\crs$ with the appropriate session ID to get the same $\crs$ (neither of them receive $t$).
			\item $R$ computes $(\pk, \sk)\gets\KeyGen(\sigma)$ and sends $\pk$ to $S$.
			\item $S$ gets $\pk$ and computes $y_b\gets\enc(\pk,b,m_b)$ and sends it to $R$.
			\item $R$ decrypts $\dec(\sk,y_\sigma)$ to obtain $m_\sigma$.
		\end{enumerate}
		\vspace{2mm}
	\end{mdframed}
	\caption{Two-Round OT from a Dual-Mode Cryptosystem.}
	\label{fig:dmmodeot}
\end{figure}

\paragraph{Security.} We see briefly how this scheme provides security.
\begin{itemize}
	\item In messy mode, $y_\sigma$ is a correct encryption of $m_\sigma$, but $y_{1-\sigma}$ is statistically unrelated to $m_{1-\sigma}$. Thus the receiver gets no information about $m_{1-\sigma}$ at all. The receiver has computational security since $\pk$ computationally hides $\sigma$.
	\item In decryption mode, the trapdoor allows the production of a public key which is statistically indistinguishable from a $\KeyGen(\sigma)$ output, so $\pk$ statistically hides $\sigma$. However, sender security is computational, since a `true' decryption key exists (the one output by $t$).
\end{itemize}

\subsubsection{Assumptions for Dual-Mode Cryptosystem}

A dual-mode cryptosystem can be constructed based on $\mathbf{DDH}$, $\mathbf{DCR}$, $\mathbf{QR}$ and $\mathbf{LWE}$. Information about the constructions is in $\cite{dualmode}$. In particular, the construction based on $\ddh$ mimics the OT of Figure~\ref{fig:DDHOT}.

\subsection{Extending the Usefulness of OT}

We now look at two techniques which allow subtle (more useful) variants of OT.

\subsubsection{Domain Extension} This technique allows us to obtain OT for $\ell$-bit strings from OT for $\secpar$-bit strings. Note that $\secpar$ is the security parameter \textbf{--} hence, it has to be a reasonable key length.

\begin{figure}[h]
	\begin{mdframed}[
		linecolor=black,
		linewidth=1pt,
		roundcorner=5pt,
		backgroundcolor=white,
		userdefinedwidth=\textwidth,
		]
		\vspace{2mm}
		\begin{itemize}
			\item Sender samples $k_0, k_1\gets\{0,1\}^\secpar$ and sends it to $\func_\mathsf{OT}$.
			\item Receiver sends $b$ to $\func_\mathsf{OT}$ and receives $k_b$.
			\item Sender sends $c_i = \enc_{k_i}(m_i)$ for each $i\in\{0,1\}.$
			\item Receiver decrypts $m_b = \dec_{k_b}(c_b)$.
		\end{itemize}
		\vspace{2mm}
	\end{mdframed}
	\caption{OT Domain Extension.}
	\label{fig:OTDomainExtension}
\end{figure}

\subsubsection{$1$-out-of-$N$ OT from $1$-out-of-$2$ OT}

For simplicity, we can assume $N=2^k$ for some $k$. Suppose that the receiver wants $m_\alpha$ for some $|\alpha|=k$.

\begin{figure}[h]
	\begin{mdframed}[
		linecolor=black,
		linewidth=1pt,
		roundcorner=5pt,
		backgroundcolor=white,
		userdefinedwidth=\textwidth,
		]
		\vspace{2mm}
		\begin{itemize}
			\item Sender samples $k_i^{b}$ for $b\in\{0,1\}$ and $i\in[k]$, and submits $(k_i^0, k_i^{1})$ to $\func_\mathsf{OT}$.
			\item Receiver sends $\alpha_i$ to the $i$th OT.
			\item Sender sets
			$$c_\beta = m_\beta\oplus \bigoplus_{i=1}^k F_{k_{\beta_i}}(\beta)$$ and sends each $c_\beta$. 
			\item Receiver decrypts $m_\alpha = c_\alpha\oplus\bigoplus_{i=1}^k F_{k_{\alpha_i}}(\alpha)$.
		\end{itemize}
		\vspace{2mm}
	\end{mdframed}
	\caption{$1$-out-of-$N$ OT from $1$-out-of-$2$ OT.}
	\label{fig:1outofNOT}
\end{figure}