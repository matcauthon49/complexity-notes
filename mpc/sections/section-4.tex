\section{OT Extension}

As we have seen, it is possible to perform OT of strings of long length using OT for strings of shorter length and a PRG. We will now answer the question of whether it's possible to perform a greater \textit{number} of OTs using a fewer number of OTs.

\subsection{IKNP OT Extension} 

In this section we will describe the IKNP OT Extension protocol from \cite{iknp}. The protocol works by extending $\secpar$ pairs of $m$-bit OT to $m$ pairs of $\ell$-bit OT. A full description is below. We note by $\mathbf{OT}^{a}_{b}$ an $a$-pairs $b$-bit OT functionality.

\begin{figure}[h]
	\begin{mdframed}[
		linecolor=black,
		linewidth=1pt,
		roundcorner=5pt,
		backgroundcolor=white,
		userdefinedwidth=\textwidth,
		]
		\vspace{2mm}
		\textbf{Parameters:} The Sender holds $m$ pairs of $\ell$-bit strings $\{(m_{i,b})\}$. The receiver holds $m$ selection bits $r=(r_1,\dots, r_m)$. $H:[m]\times\{0,1\}^{\secpar}\rightarrow\{0,1\}^\ell$ is a random oracle.\\

		\textbf{Protocol:}		 
		\begin{itemize}
			\item Sender selects $s\rgets\{0,1\}^\secpar$ and sends it to $\mathbf{OT}^{\secpar}_{m}$ (as the receiver).
			\item Receiver selects a random matrix
			$$T=
			\begin{pmatrix}
				T_1 & T_2 & \hdots & T_\secpar
			\end{pmatrix}_{m\times \secpar}=\begin{pmatrix}
			T^1 \\ T^2 \\ \vdots \\ T^m
		\end{pmatrix}_{m\times \secpar}$$ and sends the inputs $\{(T_i, r\oplus T_i)\}_{i\in[\secpar]}$ to $\mathbf{OT}^{\secpar}_{m}$ (as the sender).
			\item Denote by $Q$ the matrix received by the sender, which is
			$$Q=
			\begin{pmatrix}
				Q_1 & Q_2 & \hdots & Q_\secpar
			\end{pmatrix}_{m\times \secpar} = \begin{pmatrix}
			Q^1 \\ Q^2 \\ \vdots \\ Q^m
		\end{pmatrix}_{m\times \secpar}$$
		\item Sender sends 
		$$(y_{j,0}, y_{j,1})=(x_{j,0}\oplus H(j,Q^j), x_{j,1}\oplus H(j, Q^j\oplus s)).$$
		\item For each $1\leq j\leq m$, receiver outputs $y_{j, r_j}\oplus H(j, T^j)$.
		\end{itemize}
		\vspace{2mm}
	\end{mdframed}
	\caption{The IKNP OT Extension Protocol.}
	\label{fig:IKNP}
\end{figure}

\paragraph{Correctness.} We will now see why this protocol works. Consider $r_i = 0$ for some $i$. It follows that $(r_i)_{j\in[\lambda]}\oplus T^i = T^i$, and so $T^i=Q^i$ (regardless of whatever $s$ was). It immediately follows that $H(j, T^j) = H(j, Q^j)$.

On the other hand, if $r_i = 1$, then $Q^i = (s \cdot r)\oplus T^j =s\oplus  T^j$. Correctness follows similarly.

\paragraph{Discussion.} The protocol provides perfect security against a semi-honest (even malicious) sender and statistical security against a semi-honest receiver. The protocol is instantiated with a random oracle, but can be instantiated with a weaker primitive called a \textit{correlation-robust hash function}.

\begin{definition}[Correlation Robustness]
	An \rarrow efficiently computable function $h:\{0,1\}^{*}\rightarrow \{0,1\}$ is said to be \textit{correlation robust} if for any polynomial $q(\cdot)$ and PPT adversary $\mathcal{A}$ there exists a negligible function $\epsilon(\cdot)$ such that
	$$|\Pr[\mathcal{A}(t_1,\dots,t_m, h(t_1\oplus s), \dots, h(t_m\oplus s))=1]-\Pr[\mathcal{A}(U_{(\secpar+1)m})=1]|\leq\epsilon(\secpar)$$
	where $m\leq q(\secpar)$. The probability is taken over random and independent choices of $s, t_i\rgets\{0,1\}^\secpar$ for $i\in[m]$.
\end{definition}

The authors also modify the scheme to achieve full security against a malicious receiver through a cut-and-choose mechanism, in which a the receiver and the sender execute $k$ instances of the (committed) protocol in parallel and the sender randomly checks whether a certain number of instances were correctly performed (and aborts otherwise).

\paragraph{Parameters.} The above protocol is secure as long as $m=2^{o(\secpar)}$. Note that if, say, $m = 2^\secpar$, then the security of the correlation-robust hash function breaks down since this allows a `repeat', i.e. $H(j, Q^j\oplus s)$ will be called twice for some value of $j$. We will see the impossibility of $2^{\Omega(\secpar)}$-pairs OT extension in Minicrypt along with other impossibility results below. This concludes that the protocol is asymptotically optimal.

\paragraph{Improvements.} \cite{vole} achieves $n$ pairs using $\log n$ OTs and the (computational) LPN (Learning Parity with Noise) assumption. \cite{softspokenot} presents a practical improvement on the protocol.

\subsection{Feasibility of OT Extension}

The protocol of \cite{iknp} is not the only OT Extension protocol. Before this, \cite{beaver1996correlated} showed that $k$ OTs can be extended to $k^c$ OTs making a non-black box use of a one way function. The protocol of \cite{iknp} can gain up to superpolynomial OTs with a black-box use of an OWF, but makes use of a Random Oracle.

\textbf{Open Problem.} Is there a protocol for (subexponential) OT extension that makes black-box use of a One Way Function without a random oracle?

\cite{lindell2013feasibility} study the feasibility of OT Extension, and prove the following results.

\begin{theorem}[Information-Theoretic OT Extension]
	\label{otowf}
	If \rarrow there exists an OT extension protocol from $n$ to $n+1$ (with security in the presence of static semi-honest adversaries), then there exist one-way functions.
\end{theorem}

This proves that OT cannot be extended information-theoretically. Note that while the protocol of \cite{iknp} does not make explicit use of an OWF, it does use a random oracle (which implies OT extension).

\begin{theorem}[Adaptively Secure OT Extension]
	If \rarrow there exists an OT extension protocol from $n$ to $n+1$ that is secure in the presence of adaptive semi-honest adversaries, then there exists an oblivious transfer protocol that is secure in the presence of static semi-honest adversaries.
\end{theorem}

This proves that any adaptive OT extension protocol involves constructing statically secure OT extension from scratch. The problem of constructing adaptively secure OT extension was solved in \cite{byali2017fast}, though their protocol uses public-key cryptography.

\begin{theorem}[Extending Logarithmic OTs]
	If \rarrow there exists an OT extension protocol from $f(n)=O(\log n)$ to $f(n)+1$ that is secure in the presence of static malicious adversaries, then there exists an OT protocol that is secure in the presence of static malicious adversaries.
\end{theorem}

This proves that any OT extension protocol that have exponential extension must make use of an exponential number of OTs itself.

\subsubsection{OT Extension Implies One-Way Functions}

We now provide a sketch of the proof of Theorem~\ref{otowf}, as given in \cite{lindell2013feasibility}.

\begin{lemma}[OT Extension with Polynomial Stretch]
	Let \rarrow $f:\mathbb{N}\rightarrow\mathbb{N}$ be any polynomially-bounded function, and let $n$ be the security parameter. If there exists a protocol $\pi$ that is an OT-extension from $f(n)$ to $f(n)+1$ that is secure in the presence of adaptive, malicious adversaries, then for every polynomial $p(\cdot)$ there exists an OT-extension protocol from $f(n)$ to $p(n)$ that is secure in the presence of adaptive malicious adversaries.
\end{lemma}

\begin{proof}
	We begin by noting that any OT-extension protocol $\pi$ can be converted into an extension protocol $\pi'$ in two phases:
	\begin{itemize}
		\item\textbf{Ideal OT}: In this phase, the parties make $f(n)$ calls to an ideal OT.
		\item\textbf{Computation}: In this phase, the parties use the OT calls from the previous phase to get $f(n)+1$ OTs.
	\end{itemize}
	The protocol to get $p(n)$ OTs is simple: first, use the OTs from the first round to get $f(n)+1$ OTs, and then \textit{reuse} them $p(n)$-many times to get $p(n)$ OTs (each invocation adds one additional OT). The formal proof proceeds by hybrid argument.
\end{proof}

We will now use this fact in order to generate two polynomial-time constructable probability ensembles that are \textit{statistically} distinguishable but \textit{computationally} indistinguishable. The existence of such ensembles is equivalent to a one-way function, shown by \cite{goldreich1990note}. In particular, Goldreich shows that the existence of pseudorandom generators is equivalent to the existence of such ensembles.

\begin{proof}[Proof Sketch of Theorem~\ref{otowf}]
	Let $\pi$ be an $n\rightarrow 2n+1$ OT-extension.
	Consider the random variables $X_0, X_1, X_0', X_1',\Sigma$, where
	\begin{itemize}
		\item $\Sigma\in\{0,1\}^{2n+1}$ is a uniformly distributed string (representing the receiver's input).
		\item $X_i, X_i'\in\{0,1\}^{2n+1}$ are uniformly distributed under the constraint that $X^i_{\Sigma^i}={X'}^i_{\Sigma_i}$ (representing the possible sender inputs \textbf{--} if $\Sigma^i$ is $b$, then $X_b$ and $X_b'$ agree on the $i$th bit, otherwise they are independent).
	\end{itemize}
	Let $\textsc{trans}^\pi(x_0,x_1,\sigma)$ be the transcript of $\pi$ on the sender inputs $x_0$ and $x_1$ and on receiver input $\sigma$ (the transcript does \textit{not} contain inputs to the ideal OT functionality, however). Define the probability ensembles
	\begin{align*}
		\mathcal{E}^0_n &= (X_0, X_1, \Sigma,\textsc{trans}^\pi(X_0,X_1,\Sigma))\\
		\mathcal{E}^1_n &= (X_0', X_1', \overline{\Sigma},\textsc{trans}^\pi(X_0,X_1,\Sigma))
	\end{align*}
	The key idea is this: these ensembles must be \textit{computationally} indistinguishable since the receiver should be completely unable to distinguish whether $X_i$ or $X_i'$ was used given the transcript, since otherwise receiver security of the OT will be violated. Similarly, $\Sigma$ should be indistinguishable from $\overline{\Sigma}$, since the sender security will be violated. Note that the first 3 elements are indistinguishable simply by construction. The non-trivial computational indistinguishability is that of the transcript.
	
	Note, however, that the transcript must be \textit{statistically} distinct. This is because the transcript \textit{has} to contain some meaningful information about the input distributions \textbf{--} the only way this is impossible is if \textit{all} information were communicated by the ideal OT that is executed as a subprotocol. But this cannot be the case, since only $n$ instances of OT were carried out; some information about the other $n+1$ has to be contained in this transcript. This provides an intuitive reason why the two are statistically indistinguishable.
\end{proof}

The full proof is contained in Section 3 of \cite{lindell2013feasibility}.

