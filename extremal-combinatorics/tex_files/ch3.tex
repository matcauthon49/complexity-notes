\subsection{Review}

Some filling-in-the-details here, in order to make sure that I'm completely understanding what's happening.

\textbf{Page 29, Prop. 2.8.} The final inequality follows from the averaging principle, but it's not as immediate as meets the eye. We basically use the fact that the elements of $\mathcal{F}$ form a partition of the vertices of $G$, and the fact that every single one of them is a clique. In particular, for any $X,Y\in\mathcal{F}$, the edges from $X$ to $Y$ are not \textit{complete}, and hence $X\cup Y$ is not a clique (also because of the independence of $\alpha$, none of the subsets are cliques either). This means that this is a complete characterization of \textit{all the cliques} in the graph. We then get that the average size of a clique is
$$\frac{1}{|\mathcal{F}|}\sum{X_i\in\mathcal{F}}|X_i| = \frac{n}{|\mathcal{F}|}$$
and it follows from the averaging principle there exists some clique with size greater than or equal to the average clique, and hence
$$\omega(G)\geq\frac{n}{|\mathcal{F}|}.$$

\subsection{Problems}

\paragraph{Question 2.1} Start with the sets $A_i$ and construct new sets $A_{i,j}=A_i\cap A_j$ and $A_i' = A_i\setminus(\cap_{j\neq i} A_j)$. For each $i$, $A_i = A_i'\cup (\cup_{j\neq i} [A_i\cap A_j])$. Then by the union bound we have
$$|A_i|\leq |A_i'|+\sum_{i\neq j}|A_i\cap A_j|$$
and it follows that 
$$\sum_{i=1}^m |A_i| \leq \sum_{i=1}^m |A_i'| + \sum_{i}\sum_{j\neq i}|A_i\cap A_j|$$
Since $|A_i\cap A_j|\leq t$, we have the latter half of the RHS to be $\leq t\binom{m}{2}$. For the former, we notice that $A_i'\cap A_j'=\emptyset$ by definition. Then $\sum_{i=1}^m |A_i'|\leq n$. We have
$$\sum_{i=1}^m |A_i|\leq n + t\binom{m}{2}.$$

\begin{center}
	\line(1,0){70}
\end{center}

\paragraph{Question 2.3} (Did this \textit{without} the hint, very happy.) 

Let $d(x)$ be the degree of $x$ in $\mathcal{F}$. Then $p(x)=d(x)^2 + (m-d(x))^2$. This is because the first condition is satisfied by all $(A,B)$ such that $x$ is in both of them, ie. exactly $d(x)$ of them, while the latter is satisfied by $(A,B)$ such that $x$ is in neither of them. We can expand this to write $p(x) = 2d(x)^2 + m^2 - 2md(x) = m^2 - 2d(x)(m-d(x))$. Now $d(x)(m-d(x))\leq m^2/4$ (which can be easily checked \textbf{--} it reaches its maximum at $d(x)=m/2$) and we have $p(x)\geq m^2-2(m^2/4)=m^2/2$.

\begin{center}
	\line(1,0){70}
\end{center}

\paragraph{Question 2.7} We argue using induction. Suppose that every $(r-1)$-partite $2$-clique free graph contains $\leq 2m^{r-1-1/2^{r-2}}$ edges. The proposition is trivially true for $r=1$.

Consider now an $r$-partite graph with more than $2m^{r-1/2^{r-1}}$ vertices. We can write any $r$-partite graph $G$ as as a subset of the cartesian product $X \times V = V_1\times\dots\times V_{r-1}\times V$. Define the sets $A_v=\{x\in X:(x,v)\in G\}$; we will apply the lemma to them. Clearly the size of $X$ is at most $m^{r-1}$, and so the size of any $A_v$ is at most $m^{r-1}$, and since $v\in V$ we have that the number of such sets is at most $m$. We calculate the average size of the sets as $m^{r-1}/m$, which is $m^{r-2}$. 

Taking $w=\frac{1}{2}m^{1/2^{r-1}}$, we can apply the lemma to find two sets $A_i$ and $A_j$ such that their intersection is of size more than
$$\frac{n}{2w^{2}}=\frac{2m^{r-1}}{m^{1/2^{r-2}}}=2m^{r-1-1/2^{r-2}}.$$
Then $A_i\cap A_j$ must contain a $2$-clique. But this $2$-clique is connected to $i$ and $j$, and hence there is an $m$-partite $2$-clique in this graph.

\begin{center}
	\line(1,0){70}
\end{center}

\paragraph{Question 2.8} This is nearly identical to that of the in-text question of chapter 1, but slightly involved. The double counting trick works.

\begin{center}
	\line(1,0){70}
\end{center}