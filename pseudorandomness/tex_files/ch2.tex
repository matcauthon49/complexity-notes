\paragraph{Problem 2.9 (Spectral Graph Theory).} Let $M$ be the random walk matrix for  a $d$-regular \textit{undirected} graph $G=(V,E)$ on $n$ vertices. We allow $G$ to have self-loops and multiple edges. Recall that the uniform distribution is an eigenvector of $M$ of eigenvalue $\lambda_1=1$. Prove the following statements. (Hint: for intuition, it may help to think about what the statements mean for the behaviour of the random walk on $G$.)

(1) All eigenvalues of $M$ have absolute value at most $1$.

\begin{proof}
	Consider any eigenvector $e$ that corresponds to any eigenvalue $\lambda$. We know that $eM=\lambda e$. Suppose that $e_i$ is the index of $e$ with the highest absolute value. We know that
	$$\sum_{j=1}^{n}e_jM_{ji}=\lambda e_i.$$
	Since $e_i$ is the element with the highest absolute value, we have that 
	$$\sum_{j=1}^{n}e_jM_{ji}\leq e_i\sum_{j=1}^nM_{ji}=e_i.$$
	It follows that $\lambda e_i\leq e_i$, implying that $\lambda\leq 1$.
\end{proof}

(2) $G$ is disconnected $\iff$ $1$ is an eigenvalue of multiplicity at least $2$.

\begin{proof}
	Suppose that there are two subsets $A,B\subseteq V$ where $A$ and $B$ are completely disconnected. We define $e_X$ for each $X\subseteq V$ to be
	$$e_X := \left(\frac{\mathds{1}_X(i)}{|X|}\right)_{i\in V}$$
	Then $e_AM = e_A$ and $e_bM = e_B$, but $e_A$ and $e_B$ are linearly independent. Thus the graph $G$ has eigenvalue $1$ with multiplicity at least $2$. For the reverse implication, suppose there is a vector $v$ with an eigenvalue $1$ where $u\neq v$. Then the subspace spanned by $\left\langle u,v\right\rangle$ is part of the eigenspace of $1$. Set $v'=c_1u+c_2v$ such that $v'$ is zero on at least one index and non-negative on the others; this is possible since $v$ and $u$ are linearly independent and each index of $u$ is non-negative. Then $v'M = c_1uM + c_2vM = v'$. We can interpret $v'$ as a probability distribution by setting $w=\frac{v'}{\|v'\|}$. Then $w$ is a probability distribution. Note that $wM = w$ since $w$ is in the eigenspace of $1$ as well.
	
	 It follows that $\lim_{t\rightarrow\infty} wM^{t}=w$. However, $w$ has some index $i$ such that $w_i=0$. This means that for arbitrarily many steps, the probability of reaching the vertex $i$ when starting from any non-zero vertex $w_j$ in $w$ is $0$. This immediately implies that there is no path from $j$ to $i$, and thus $G$ is disconnected.
\end{proof}

(3) $G$ is bipartite $\iff$ $-1$ is an eigenvalue of $M$.

\begin{proof}
	We order the vertices in $M$ as the vertices in $A$ first and the vertices in $B$ second. Then consider the vector 
	$$e := \left(\frac{{\chi_A(i)}}{n}\right)_{i\in V}$$
	where $\chi_A$ is $-1$ if $i\in A$ and $1$ otherwise. Then this is an eigenvector with eigenvalue $-1$. Now suppose that $-1$ is an eigenvalue of $M$, with corresponding eigenvector $e$. Then $eM=-e$ and $eM^2 = e$. Note that $M^2$ is also a random-walk matrix for a different undirected multigraph $G^2$, the graph formed by connecting $(u,v)\in G^2\iff \exists t:(u,t),(t,v)\in G$. Then $e$ is an eigenvector of $M^2$ with eigenvalue $1$. It follows that either $e=u$ or $e\neq u$. However $e=u$ is not possible since otherwise $eM=e\implies e=\mathbf{0}$, which is obviously false. Thus $e\neq u$ and so $G^2$ is disconnected. This means there exist sets $A, B\subseteq V$ such that there is no even-length walk from $A$ to $B$. In particular, this implies that there is no odd-length cycle in the graph (since such a cycle would `force' an even-length walk from $A$ to $B$ assuming that $G$ were connected, which it is). This means that $G$ is bipartite.
\end{proof}