\documentclass{tufte-book}
%\usepackage{lipsum}

\usepackage{amsmath}
\usepackage{amssymb}
\usepackage{amsfonts}

\newenvironment{loggentry}[2]% date, heading
{\noindent\textbf{#2}\marginnote{#1}\\}{\vspace{0.5cm}}

\begin{document}
	
	
	\begin{loggentry}{2024-Feb-14}{New Theory Frontiers}
		Had a conversation with Jiayu in the morning, in which we discussed that we would not be working on things for a couple weeks, which gives me some time to think and retread on some of the problems I'm interested in solving. What I need to do is thoroughly go through some basic algorithmic and probabilistic problem-solving techniques from a mathematical standpoint. I've found a few resources towards this goal:
		\begin{itemize}
			\item The book 'Extremal Combinatorics.' I will try to go through the first section of this book with good writing.
			\item Ryan O'Donnell's Probability and Computing lecture notes from 2009, which are probably worth going through as well.
			\item Salil Vadhan's course 'A Theorists Toolkit' from 2005, which consists of 12 lectures. Will probably go through them at some point.
			\item Finally, Babai's notes on 'Linear Algebra methods in Combinatorics.' Will refer to them if I ever need them.
		\end{itemize}
	
	Anyway, I then read chapter $1$ of the book on some basic counting. Did the binomial theorem and some definitions of factorials. This is slightly interesting:\\
	
	\textbf{Lemma.} $\sum_{i=0}^k\binom{n}{i}\leq\left(\frac{en}{k}\right)^k$.\\
	
	Proof uses the identity $1+t<e^t$. Factorial approximations may be useful; the stirling formula is supposedly important.
	$$n!=\left(\frac{n}{e}\right)^n\sqrt{2\pi n}e^{\alpha_n}$$
	where $1/(12n+1)<\alpha_n<1/12n$.
	
	Did the stars and bars counting method and some stuff related to partitions.
	
	Turan's number is quite interesting here, because I remember it being defined only in the context of graphs. An intuitive explanation for this is as follows: the number $T(n,k,l)$ with $(l\leq k\leq n)$ is the smallest number of $l$-element subsets of an $n$ element set $X$ in a way such that every $k$-element subset of $X$ contains at least one of those sets. Think of it this way: we have some set $X$ of size $n$. Then we can divide it into $l$-element subsets. Suppose we pick a $k$-element subset; assuming that we have enough $l$-element subsets, $k$ will necessarily contain one of them. Clearly, if $l$ is simply $\binom{n}{l}$, this is true, but we can choose less of them as well. The smallest number we can choose is $T(l,k,n)$.
	\end{loggentry}

\end{document}